\subsection{Benchmarking}
The process of analyzing the sound files in the field of soundscape ecology is quite painful as it stands. Currently, our sponsor uses wav format sound files of size 1.29GB, each a recording of ten minutes long. Some datasets used include over one hundred sound files, which amounts to over 100GB of sound files to process. It is important to know the length of time needed to process these files locally based on the end user\textquotesingle s hardware. The following are the results of NDSI index analysis on a single sound file of size 1.29GB, and the respective hardware involved. As a side note, it seems important to add that in order to run analysis on sound files in R, each wav file must be converted to a Wave object in R to be used in the index processes. This alone has proven to be a lengthy process on lower end systems.\\

\noindent\textbf{Intel i5 6500 @ 3.2GHz / 16GB RAM}\\
The analysis took 686.64 seconds to process a single 1.29GB wav sound file. That comes out to 1.29GB in eleven minutes and forty four seconds. For a data set comprised of one hundred 1.29GB sound files, that would estimate to around nineteen hours for 129GB of sound files.\\

\noindent\textbf{Intel i5 5200U @ 2.2GHz / 8GB RAM}\\
The process of converting this single wav file to a Wave object in R actually managed to freeze this system for around 30 minutes, before any analysis even began. Then after 315.4 seconds, a little over five minutes, the processing stopped with an error reading "cannot allocate vector of size 1.3GB." This was an unexpected result, raising questions as to possible user minimum required hardware specs. Thus, with this hardware, processing is not even feasible.\\

\noindent\textbf{Conclusions}
