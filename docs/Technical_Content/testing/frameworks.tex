\subsubsection{Frameworks}
We will use a combination of frameworks in order to test our product. These frameworks include Chai, Mocha, Spectron, and Mockgoose. Each of these frameworks is tried and true, known well in the JavaScript development community for their ease of use in testing in Node.js environments.

\paragraph{Chai} \mbox{}\\[\paragraphheaderspace]
Chai is an assertion library purposed for behavior-driven and test-driven development in Node.js. We will use Chai as Promised, which extends Chai, for asserting facts about JavaScript promises. Thereby, we can transform any Chai assertion into one that acts on a promise. Combining Chai with Spectron will provide a powerful tool that allows us to use JavaScript promises.

\paragraph{Mocha} \mbox{}\\[\paragraphheaderspace]
Mocha is a feature-rich JavaScript test framework running on Node.js and in the browser. Mocha tests run serially, allowing for flexible and accurate reporting, while mapping uncaught exceptions to the correct test cases. Mocha allows us to use Chai as our assertion library. It also supports asynchronous testing, which can be useful when running tests with large amounts of data inputs.

\paragraph{Spectron} \mbox{}\\[\paragraphheaderspace]
Spectron is an open source framework for writing automated tests for Electron apps. It allows developers to navigate web pages, simulate user input and run dedicated tests. It also simplifies the use of other testing frameworks by making the setting up, running, and tearing down of Electron environments easier. Developed by GitHub, it is the dedicated testing framework for Electron, and therefore, it is the first choice for writing test cases for our Electron-based front end application.\par
The following example code finds the application, checks to see if there is a visible window containing the title \lq My App\rq, and finishes by logging any errors:

\begin{javascriptcode}
// A simple test to verify a visible window is opened with a title
var Application = require('spectron').Application
var assert = require('assert')

var app = new Application({
  path: '/Applications/MyApp.app/Contents/MacOS/MyApp'
})

app.start().then(function () {
  // Check if the window is visible
  return app.browserWindow.isVisible()
}).then(function (isVisible) {
  // Verify the window is visible
  assert.equal(isVisible, true)
}).then(function () {
  // Get the window's title
  return app.client.getTitle()
}).then(function (title) {
  // Verify the window's title
  assert.equal(title, 'My App')
}).then(function () {
  // Stop the application
  return app.stop()
}).catch(function (error) {
  // Log any failures
  console.error('Test failed', error.message)
})
\end{javascriptcode}

\paragraph{Mockgoose} \mbox{}\\[\paragraphheaderspace]
Mockgoose is database mocking framework that allows for an in-memory MongoDB database to be spun up when calls to Mongoose (our MongoDB wrapper) are made. Mocking a database allows the development team to use a machine\textquotesingle s memory to run tests that make calls to the database, meaning that these tests have no persistence in any real database. This effectively isolates unit and integration tests from any of the problems that might come up as a result of persistent data left behind in the database from previous use.
