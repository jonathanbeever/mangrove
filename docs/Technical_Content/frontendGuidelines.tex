\section{Front End Design Guidelines}
When working on the user interface, our team mostly tried to adhere to a consistent set of principles. This list was adapted slightly from two separate manifestos to fit the purposes of our project.\par

\begin{itemize}
\item \textbf{The Simplest Option is Usually the Best:} By using what is natively supported by the tools and libraries we have already introduced into the project, the code base can become much simpler to understand amongst team members.
\item \textbf{Reduce the Amount of Moving Parts:} In addition to limitations of dependency introducuction, dependencies should also be reduced as each one can act as a point of failure.
\item \textbf{Understand the Business:} At the end of the day, the code we write only matters in so much as to how it will serve our users (i.e. Dr. Beever and other soundscape ecology researchers). Therefore, it is imperative to have their best interests in mind when working on the end product.
\item \textbf{Do Not Design Systms around Edge-cases:} By simple definition, the amount of times edge-cases occur is insignificant. As a result, the design should first seek to reflect the majority use case and only after should edge cases be adddressed.
\item \textbf{Do Not Make Decisisions Based on Anecodtal Evidence:} By simple definition, anecdotes are generally not representative of reality; they simply reflect the experience of one, single person. So, when proposing a change to the design, data should be collected (generally through testing) to support it, rather than simply listening to the story of a single use case.
\item \textbf{Expect and Acccommodate Change:} The original proposal for this project was intentionally left open ended. It was expected of us to listen to user feedback and think of novel ways to solve some of the problems that the field of soundscape ecology currently faces. Therefore: we must be ready to change the goals and strategies of our endeavors if we wish to succeed.
\item \textbf{Keep Users in Control:} By ensuring that the user interface remains as self-explanatory as possible, users can feel in control of their environment. When users are in control of their environement, they feel more comfortable and are less prone to performing incorrect or irrelevant interactions.
\item \textbf{One Primary Action Per Screen:} By limiting one primary action for each of our pages, users won't become overwhelmed with the many interactions they can make with the application. This was our goal when the front end split into three primary pages: Queue, Catalog, and Settings.
\end{itemize}
