\subsubsection{Get Job}
Retrieve the details of an existing job by \codesnip{jobId}.

\paragraph{GET} \mbox{}\\[\tabularheaderspace]
\begin{htmlcode}
https://<base-url>/api/jobs/:jobId
\end{htmlcode}

\paragraph{Header} \mbox{}\\[\tabularheaderspace]
\begingroup
\renewcommand{\arraystretch}{\cellpaddingvertical}
\begin{tabular}{| m{\fieldcolwidth} | m{\typecolwidth} | m{\desccolwidthlg} |}
  \hline
  \reqhead{Field}
  & \reqhead{Type}
  & \reqhead{Description}
  \\ \hline

  \codesnip{Content-Type}
  & String
  & \codesnip{"application/json"}
  \\ \hline
\end{tabular}
\endgroup

\paragraph{URL Parameters} \mbox{}\\[\tabularheaderspace]
\begingroup
\renewcommand{\arraystretch}{\cellpaddingvertical}
\begin{tabular}{| m{\fieldcolwidth} | m{\typecolwidth} | m{\desccolwidthlg} |}
  \hline
  \reqhead{Field}
  & \reqhead{Type}
  & \reqhead{Description}
  \\ \hline

  \codesnip{jobId}
  & String
  & The \codesnip{jobId} of the job for which details are desired.
  \\ \hline
\end{tabular}
\endgroup

\paragraph{Example Request: GET} \mbox{}\\[\tabularheaderspace]
\begin{htmlcode}
https://<base-url>/api/jobs/job-1a-uuid
\end{htmlcode}

\paragraph{Response Body Fields} \mbox{}\\[\longtableheaderspace]
\begingroup
\renewcommand{\arraystretch}{\cellpaddingvertical}
\begin{longtable}{| m{\fieldcolwidth} | m{\typecolwidth} | m{\desccolwidthlg} |}
  \hline
  \reqhead{Field}
  & \reqhead{Type}
  & \reqhead{Description}
  \\ \hline

  \codesnip{jobId}
  & String
  & A unique identifier for the job created.
  \\ \hline

  \codesnip{type}
  & String
  & The type of metric to be run on the job. Possible values: \codesnip{"aci"}, \codesnip{"adi"}, \codesnip{"aei"}, \codesnip{"bi"}, \codesnip{"ndsi"}, \codesnip{"rms"}.
  \\ \hline

  \codesnip{input}
  & String
  & The \codesnip{inputId} of the input to be analyzed for metrics.
  \\ \hline

  \codesnip{jobSpec}
  & String
  & The \codesnip{jobSpecId} of the jobSpec to be used for the job.
  \\ \hline

  \codesnip{author}
  & String
  & The \codesnip{userId} of the user who made the job request.
  \\ \hline

  \codesnip{creationTimeMs}
  & Number
  & The time of the job's creation, listed in milliseconds since the Unix epoch.
  \\ \hline

  \codesnip{status}
  & String
  & The status of the job. Possible values: \codesnip{"queued"}, \codesnip{"processing"}, \codesnip{"finished"}, \codesnip{"failed"}, \codesnip{"cancelled"}.
  \\ \hline
\end{longtable}
\endgroup

\paragraph{Example Response Body} \mbox{}\\[\jsoncodeheaderspace]
\begin{jsoncode}
{
  "jobId": "job-1a-uuid",
  "type": "aci",
  "input": "input-1-uuid",
  "jobSpec": "job-spec-a-uuid",
  "author": "user-1-uuid",
  "creationTimeMs": 1546318800000,
  "status": "finished"
}
\end{jsoncode}
