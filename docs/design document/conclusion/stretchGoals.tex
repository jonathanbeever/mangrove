\subsection{Stretch Goals Success and Difficulties}
As outlined in the Milestones report of this document, most of the stretch goals of this project were not reached. While stretch goals are set to be above and beyond goals, we aimed to fulfill them and even expected to in some capacity. The purpose of this section is to outline these stretch goals and the successes and difficulties faced during development. Additionally, this section serves to provide guidance to future work as to how to accomplish these stretch goals.\\

\subsubsection{Machine Learning}
Machine learning as it pertains to sound files is a highly difficult task in itself. Not only is it technically difficult, it also requires a much higher processing power than other forms of ML. During our own development, one model took around 7 hours to train. While libraries exist for ML with sounds, the most promising one only works in real time, in that it listens as sounds are made to make predictions. This isn\textquotesingle s technically what we were aiming to include with Mangrove, it may be useful in the future should Mangrove go mobile in some capacity.\par
One approach that was going to be taken for ML was using index outputs as features to predict if the recording contained a bird. We would be using the sound files included in the Cornell library purchased for this project, however we also needed sound files that were not birds. Ultimately, the issue of file formatting was introduced, and simply converting mp3 to wav did not allow the indices to be run on the converted files. This effectively ruined this approach. In order for this to work, the files containing the bird sounds would need to be originally recorded in the wav format, along with the non bird sounds. The man power to do this recording is simply not available in the scope of this project.\par
OT SECTION HERE

\subsubsection{Collaboration Abilities}
We, along with our sponsor, wanted to include some sort of collaborative abilities with Mangrove. The purpose of this was crowd sourcing data surrounding soundscape research, to see what kind of sounds and data were coming out of different places around the country. In addition, intra facility research collaboration was intended, and is elaborated on more in the next section.\par
The backend has been designed with this functionality in mind, and is primed for future work to be done to implement collaboration. However as development continued in our time working on Mangrove, more features pertaining to the core functionality of the program were thought of and implemented. Features like audio playback, data exporting, and ensuring that the graphs were correctly presenting data felt paramount to the core of this project, and were taken to before collaboration was designed, at least in the frontend.\par
The vision of collaboration on a grand scale like we intended included an AWS backend for storing user data that can be accessed by anyone using the service. The users could choose which data to upload for the sake of open source data and research. A chloropleth map was going to be included as a heat map of where the most research and sounds were coming from through out the country. Our sponsor Dr. Beever is very interested in crowd sourcing data and open sourcing both research and this program for anyone to use, so collaboration abilities seem the next step in the future of Mangrove.\\

\subsubsection{Research Groups}
Our sponsor mentioned that at places like Purdue University, there are large research groups in the field of soundscape ecology. This sparked the idea of allowing for research groups to be created in Mangrove, aiming to help organize group sound files and data by author. This is a type of collaboration, but in a more private way. A user could create a research group for themself, becoming an admin. From there, they could add other users to their research group. Doing so would allow for a few permissions between the members of this research group. Assuming the group is using a central dedicated server for storing and processing their sound files and results from Mangrove, any group member could access these sound files that belonged to the group. Additionally, any group member could access the results from jobs run by other group members. In the client, when viewing input files and jobs, any member can see the author of both the sound file and the job.\par
This feature along with the global collaborative abilities seem central to the future of Mangrove. We hope that as soundscape ecology becomes a bigger area of research, Mangrove becomes the core data organization and processing software for researchers. For this to happen, research group abilities along with user account abilities must happen.\\
