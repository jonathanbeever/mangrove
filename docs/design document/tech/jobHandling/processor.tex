\subsubsection{Job Processor}
In order to process an input audio file using the specification provided in a given job, all information for that job is passed by the job queue into the job processor. The job processor is where all files are analyzed using a given soundscape ecology index, with any parameters for use in the analysis.\par
The job processor depends almost exclusively on code written in the R programming language, and the server (written in Node) interfaces with R using an NPM package called \codesnip{r-script}. This allows the results of each R-processed job to be passed from an R script execution back into the server. The job processor handles ACI, ADI, AEI, BI, and NDSI jobs using a version of the \codesnip{soundecology} package enhanced by the Mangrove team, while RMS jobs are handled without the use of the \codesnip{soundecology} package, using a much simpler R script.\par
Once files are processed by the job processor and results are obtained, they are passed back to the Node side of the server through \codesnip{r-script}, sanitized, and stored directly in the database as the \codesnip{result} field of the particular job that was run. From here, all information regarding the job, including its result, can be accessed via the application programming interface, used directly by Mangrove\textquotesingle s front end client.
