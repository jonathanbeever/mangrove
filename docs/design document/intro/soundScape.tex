\subsection{Soundscape Ecology}
In an ever growing world, the concern for how human actions affect the environment is growing just as rapidly. Our day to day lives have a massive impact on the natural world around us. From our bustling cars to our crowded construction projects, the sounds of our growth can span large stretches of land past the urban landscapes. With these concerns in mind, it is of ever growing importance that we humans research how this noise pollution impacts life around us. For just as a loud sound can disturb life in a quiet neighborhood, human noise can likewise disturb neighboring ecosystems. The field of soundscape ecology focuses on matters such as these, quantifying our auditory footprint and hypothesizing how to best reduce it.\par
One intuitive example that arises in much of the field\textquotesingle s literature relates to the mating patterns of birds. When urban development springs up near previously undisturbed habitats, the noise pollution created by urban life complicates the breeding process. The calls of male birds cannot easily be heard by the females, and so the population dwindles. As one might expect, the same principle applies to predators trying to detect their prey. In either of these cases, it can be seen that human sound has great potential to negatively impact animal life, which may in turn cause large scale ecological damage.\par
With a careful ear listening closely to the sounds of natural life, we can attempt to discern the state of an ecosystem over a period of time. However, this can only get us so far. When researching the health of these natural environments over the course of years, it becomes too difficult for an individual or group to reliably oversee hundreds to even thousands of hours of natural sounds. As a result, soundscape ecologists have recognized the necessity of quantifying the health of an ecosystem using algorithmic measures.