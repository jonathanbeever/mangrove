\begin{center}
IV. Mangrove Features
\end{center}
\begin{flushleft}
\setlength{\parindent}{0.125in}
The goal of our project is to provide a tool to researchers that will present data obtained by the R \codesnip{soundecology} package in a more accessible format. Mangrove uses the existing  algorithms to produce data from audio recordings and turn it into meaningful visualizations. Through Mangrove\textquotesingle s user interface, sound files along with specifications on the R algorithms are added to a queue and processed by the server. Users are given a visual representation of the progress of the analysis. This feedback greatly improves the experience of running these algorithms when compared to R Studio or the command line, which does not give progress updates until analysis of all files is complete.\par
Multiple visualizations are available for each index to accommodate for the various questions  researchers have when looking at data from particular locations. Biodiversity changes can be seen on graphs that show results based on time. Multiple data sets can be compared visually to give insight on the cause of the decline in health of a population. Outliers can be identified and compared to the corresponding audio segment with an on-screen player. These features will aid researchers in gaining insight from their data much more than the large set of numbers given by the \codesnip{soundecology} package alone.\par
\pagebreak[2]
\end{flushleft}
