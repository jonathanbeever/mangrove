\begin{center}
I. Soundscape Ecology
\end{center}
\begin{flushleft}
\setlength{\parindent}{0.125in}
Soundscape ecologists use the mathematics of waves in order to derive meaning from sound. This section provides a background on our understanding of sound, including the basics of waves, representation of waves, and Laplace transforms.\par

When we take a closer look at sound waves, certain similarities amongst all waves start to appear. These characteristics can be used to describe waves and how they interact with their environment. The most prominent feature is that simple tones are comprised of simple sinusoidal waves. These are generally graphed as the force of pressure over time. Once graphed, the wavelength of a wave can be measured. This measurement shows the distance between the peak forces over time. One wavelength can be understood as one cycle of the wave. The amount of cycles that appear per time unit can be described as the frequency of the wave. A higher frequency wave correlates to a higher ``pitch'' when heard. Additionally, the higher the wave peak, the louder the sound is. The ``loudness'' of a sound is measures in decibels (dB), which are measured in base 10 units. This measurement system is a reference system, with 0 being the minimum value that a healthy human can hear.\cite{villanueva}\par

Sound waves can vary greatly in frequency, ranging from lower pitches around 20 Hz up to super sonic sounds at 60 kHz. The frequency at which a sound is produced greatly influences how it is affected by the environment through which it travels. High frequency sounds have a tendency to be absorbed by obstructions in the environment, such as leaves, and obstructions by nature limit the range at which high frequency sounds can travel. In contrast lower frequency sounds have more ``flexibility'' and can move through more obstructions in an environment, allowing them to travel much further than high frequency sounds. This leads to very different sound profiles depending on the geographical location in which observations take place.\par

We use tools called spectrograms in order to visualize these discrepancies within geographical locations. It is important that, when conducting research in soundscape ecology, we keep in mind how the physical environment will block the more delicate higher frequency sounds from the observer. This is usually counteracted by using multiple recording devices placed in equidistant patterns that cover the area of interest. Using this technique allows recording devices to pick up on sounds that obstructed devices might miss.\par

The mechanism by which complex sound waves are turned into bands of frequency is central to how soundscape ecology is conducted, and so it is beneficial to have at least a basic understanding of the process. This is where the Fourier transform comes into play. Using the Fourier transform, we can find the ``loudness'' and ``pitch'' of these complex waves, and use these values to come up with meaningful data from our sound files. This is where the mathematical indices come into play.
\end{flushleft}
