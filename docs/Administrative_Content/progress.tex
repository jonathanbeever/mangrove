\subsection{First Semester Progress and Successes}
\subsubsection{Tools for Success}
\subsubsection{Milestone Progress}
\paragraph{Phase 0 - Requirement Gathering and Initial Design} \mbox{}\\[\paragraphheaderspace]
The first milestone in this section, gathering requirements has been completed throughout various meetings with our team, sponsor, TAs and professor. Discussions with our sponsor have clarified what it means for our project to be successful and determined which features should be considered stretch goals. In meeting with our professor, the importance attempting to implement stretch goal features has been discussed. Are team has agreed on pursuing machine learning stretch goals and we have allocated time for this as shown in the milestones section.\par
The aspects of project design, including the database, API/backend and frontend have been through various iterations and have continued to improve throughout the first semester. A more detailed account of this process can be found in the design iterations section of this document. We have settled on most major design decisions at this time, but will continue to implement design improvements in the second semester if they arise.\par
\paragraph{Phase 1 - Research and Prototyping} \mbox{}\\[\paragraphheaderspace]
We have made a good amount of progress prototyping our server and client applications compared to the milestone dates set at the start of semester one. Progress on the server side application includes some API requests. A request sent through Postman to create a new job will successfully update a local instance of our database. The backend team will continue to develop the requests outlined in the Application Programming Interface section.\par
Prototyping of the client application is also off to a good start. The application runs on Electron in a development environment and the front end team has made progress on pages of the application that are not related to group collaboration. The client prototype application has a navigation panel for each of the pages that have started to be implemented, catalog, job queue and settings.\par
Progress on the catalog page includes job filtering and searching, which is almost completed to function with sample data. On this page, progress on D3 visualisations for results has also been made for some indices and data structured the way it will be in the database can be viewed in D3 graphs. By the end of December, these two features should be working together by showing job results of any sample job searched. The job queue page, where a user will start new jobs currently includes UI selection of indices and parameters. The next step on this page is file input selection for jobs and after this is implemented, requests can be sent to the API to make a new job. An outline of all of the input components on the settings page has been completed. Starting to work on the functionality of this page will require API requests related to users and groups to be created first.\par
Prototyping of the AWS backend has not started yet. Our team decided to complete all aspects of the local environment prototypes before moving on to the paid AWS remote version of the prototype.\par